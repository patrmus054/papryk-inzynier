\section{Podsumowanie i wnioski}
Pisząc naszą pracę chcieliśmy zgromadzić jak największą ilość wiedzy oraz dobrych praktyk, dotyczących nowoczesnych metod tworzenia oprogramowania. Z tego też powodu pracę otworzyliśmy rozdziałem wprowadzającym, w którym przybliżyliśmy jak istotną rolę w dzisiejszym przemyśle informatycznym odgrywa automatyzacja procesów. Dużą część naszej pracy stanowi teoria, ponieważ uważamy, że zrozumienie idei jest pierwszym i najistotniejszym krokiem, dzięki któremu nasza praca przy automatyzacji będzie skuteczna.

Automatyzacja jest stosowana powszechnie w praktycznie każdym zespole programistycznym. Z powodu jej powszechności i dostępności materiałów do nauki coraz częściej implementowana jest też przez programistów w prywatnych projektach, ponieważ widzą oni wartość dodaną, która ona wnosi. 

My również zauważyliśmy takie korzyści, dlatego podczas pracy zdecydowaliśmy się na implementację zasady CI/CD, dzięki której nasz plik \LaTeX po opublikowaniu zmian w repozytorium automatycznie budował się i publikował w formacie .pdf na stronie internetowej w formie łatwej do czytania. 

\begin{figure}[htbp]
    \centering
    \includegraphics[width=12cm]{images/podsumowanie.png}
    \caption{Budowanie nowych wersji naszej pracy, źródło: własne}
    \label{fig:podsumowanie}
\end{figure}

Podejście CI/CD pozwala programistom na efektywne wykonywanie swoich obowiązków. Eliminowane są zbędne przestoje w pracy, problemy wynikające z regresji w kodzie, a same nowe funkcjonalności publikowane są szybciej. Ciągłe testowanie wzmacnia pewność siebie programisty, nie musi się on martwić, że jego zmiany spowodują problemy dla reszty członków zespołu, w najgorszym wypadku jego zmiany nie zostaną zaakceptowane. 

Uważamy, że temat naszej pracy można jeszcze zdecydowanie pogłębić. Możliwa byłaby analiza innych platform programistycznych oraz metod tworzenia oprogramowania.  Dziedzina ta bardzo szybko się rozrasta i uważamy, że warto interesować się zagadnieniami automatyzacji w programowaniu, ponieważ może się to okazać opłacalne w przyszłości.  